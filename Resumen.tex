\thispagestyle{empty}
\chapter*{Resumen\markboth{Resumen}{Resumen}}


En el contexto del entrenamiento médico, los simuladores de realidad virtual representan un método seguro para el entrenamiento de habilidades cognitivas y no cognitivas en un entorno controlado y seguro. Es importante que un profesional sanitario en formación se enfrente a la mayor cantidad de casos y variaciones anatómicas posibles.

\todo{en pasado o presente?}
Esta tesis ha sido desarrollada dentro del contexto del proyecto \emph{RASimAs}, financiado por el Séptimo Programa del Marco de la Unión Europea. El objetivo principal es el desarrollo de herramientas que faciliten el entrenamiento y la práctica de la anestesia regional. Para cumplir con este objetivo se ha propuesto dos sistemas: un entrenador de realidad virtual llamado \emph{Regional Anaesthesia Simulator (RASim)} y un asistente en quirófano llamado \emph{Regional Anaesthesia Assistant (RAAs)}.
Con el objetivo de enfrentar a los médicos a la mayor cantidad de variabilidad anatómica posible, en este proyecto se ha propuesto desarrollar un entorno integrado de generación de pacientes virtuales, que permita crear una base de datos de modelos anatómicos que pueda ser utilizado por el simulador \emph{RASim}. En concreto, la tarea de transformar la postura del paciente virtual a la posición requerida por el procedimiento de anestesia regional dentro de este entorno ha sido designada al grupo \emph{GMRV} de la \emph{Universidad Rey Juan Carlos} y, por tanto, es la motivación principal de esta tesis.

En este trabajo, se presenta una nueva técnica que permite transformar pacientes virtuales desde la posición original a la postura requerida por el entrenamiento del procedimiento. Esto es posible aunque los modelos anatómicos se encuentren incompletos o falten sus descripciones mecánicas. Además, ya que el usuario supervisará la deformación que se aplicará al paciente virtual, el sistema debe tener tasas de refresco interactivas. Para cumplir con estos requisitos, se ha seguido un enfoque geométrico desarrollando una variación del cauce de animación esqueletal debido a que los métodos basados en física presentan una serie de problemas. Estos requieren caracterizar mecánicamente los tejidos que se van a simular los cuales no siempre se encuentran disponibles. Además, los métodos basados en modelos físicos se centran en conseguir deformaciones precisas, resultando en un alto grado de complejidad que impide conseguir tasas de refresco interactivas. Frente a los métodos basados en física, se ha optado por utilizar una técnica geométrica que proporciona soluciones plausibles que el usuario pueda interpretar como reales.

Partiendo de la piel y el tejido óseo del paciente virtual, se genera un campo de desplazamientos continuo en el interior del paciente virtual que se utilizará para transformar sus estructuras internas. Las operaciones más costosas se han delegado a un proceso previo que generará toda la información necesaria para que el usuario pueda seleccionar la postura del paciente virtual interactivamente. Adicionalmente, se ha propuesto un refinamiento opcional basado en un método físico que intenta conservar el volumen del modelo anatómico. Con el objetivo de validar la hipótesis por la cual un algoritmo geométrico puede generar nuevas posturas de un paciente virtual junto con sus tejidos internos para ser utilizadas en el contexto del entrenamiento de un procedimiento médico, se han propuesto dos casos de uso. 

En primer lugar, el algoritmo propuesto se ha integrado en el entorno de generación de pacientes virtuales, permitiendo animar y adaptar al profesional médico los modelos anatómicos generados. Además, con la finalidad de comprobar si los pacientes virtuales son útiles para  el entrenamiento del procedimiento de RA, se ha contribuido en la creación del módulo Courseware. Esta plataforma de aprendizaje donde el usuario podrá practicar y desarrollar sus habilidades no cognitivas gestiona todos los componentes del simulador y se encarga de implementar las tareas de entrenamiento. Por problemas de precisión de los dispositivos hápticos, no se ha podido realizar una evaluación clínica del simulador. 



En segundo lugar, se ha desarrollado un simulador de radiología diagnóstica gracias a la librería \emph{gVirtualXRay} en colaboración con \emph{Dr. Franck P.Vidal}. En este procedimiento, el médico debe posicionar al paciente y configurar la máquina de rayo X de manera que la región anatómica objeto de estudio sea adecuadamente capturada. El algoritmo propuesto permite modificar la posición de un paciente virtual interactivamente, y así probar distintas proyecciones, mientras que la librería \emph{gVirtualXRay} permite  obtener imágenes de rayos X simultáneamente. Se ha realizado una validación del simulador para confirmar su utilidad como herramienta adicional a las técnicas clásicas de aprendizaje.

